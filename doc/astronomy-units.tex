\documentclass[12pt, letterpaper]{article}
\usepackage{hyperref}
\title{Astronomy Distance Units Explained}
\date{July 2023}
\begin{document}
\maketitle
\begin{abstract}
    \noindent This document provides an "explain like I'm five" explanation for three astronomy distance units:
    \noindent the astronomical unit (AU), the parsec, and the light-year. Refer to Table
    \noindent \ref{table:data} for the conversions between different distance units and meters.
\end{abstract}

\section*{Astronomical unit}  % Unnumbered section
The astronomical unit (AU) has traditionally been the average distance between the Sun and the Earth, but has now been defined as exactly \href{https://www.iau.org/static/resolutions/IAU2012_English.pdf}{149 597 870 700 m}.

\section*{Parsec}

\section*{Light-year}
The light year is the distance light travels in a year, with a year being defined as \href{https://web.archive.org/web/20070216041250/http://www.iau.org/Units.234.0.html}{365.25 days}. So,
because the speed of light is defined as exactly 299 792 458 m/s, and there being 86 400 seconds in
a day, there is exactly 9 460 730 472 580 800 meters in a light-year.



\begin{table}[h!]
\centering
    \begin{tabular}{||c c||}
        \hline
        Distance unit & Meters \\ [0.5ex] 
        \hline\hline
        Astronomical unit & 1 \\
        \hline
        Parsec & 1 \\
        \hline
        Light-year & 9460730472580800  \\
        \hline
        Gigameter & 1000000000 \\ [1ex] 
        \hline
    \end{tabular}
\caption{Conversion to meters of different distance units}
\label{table:data}

\end{table}

\end{document}
